\section{Conclusiones}

La integración de fotometría terrestre y espacial en este trabajo ha permitido validar un protocolo completo de reducción de datos y análisis exoplanetario. La observación realizada desde la UAM confirma la solidez del proceso de calibración instrumental para diagnosticar la magnitud aparente de cuerpos celestes, resolviendo la magnitud aparente de TOI-4010, $m_{V,\text{exp}}$, en consonancia con la bibliografía. Para los datos de archivo, el \textit{pipeline} computacional, basado en el algoritmo \textit{Box Least Squares} (BLS) y técnicas de \textit{Pixel Level Decorrelation} (PLD), consiguió estimar el radio y semieje mayor para todos los planetas de corto periodo de TOI-4010 con una distancia inferior a $1\sigma$ respecto al valor de referencia. Para ambos orígenes de datos, \textit{Target Pixel Files} (TPFs) y \textit{Light Curve Files} (LCFs), se obtuvo un error relativo medio inferior al $5\%$. El uso de TPFs o LCFs resultó indistinto para determinar el semieje mayor, pero sí que mostró diferencias en cuanto a la obtención del radio planetario, arrojando LCFs resultados más certeros en promedio.

A nivel estadístico, la eficacia del $84.8\%$ para determinar periodos con una precisión del $99\%$ alcanzada por el \textit{pipeline} automatizado en sistemas de periodo corto $(T \in [0.5, 20] \text{ días})$ ratifica la viabilidad de búsquedas ciegas masivas, a pesar de las detecciones no coincidentes y los \textit{alias} armónicos intrínsecos al algoritmo. La congruencia de la muestra analizada con los modelos de Chen \& Kipping y Otegi et. al en el diagrama MR reafirma la precisión del \textit{pipeline}, evidencia el sesgo computacional hacia cuerpos de gran radio, condicionado por la relación $\text{SNR} \propto \delta \propto R_p^2$, y permiten intuir la composición planetaria sin necesidad de recurrir a espectroscopía.
 Finalmente, la identificación de TOI-715 b y TOI-6002 b en regiones de habitabilidad conservativa y de Venus reciente, respectivamente, enfatiza la necesidad de búsquedas masivas de exoplanetas para poder encontrar mundos susceptibles de ser habitables, dada su baja frecuencia de aparición.