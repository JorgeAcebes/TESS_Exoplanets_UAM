\section{Introducción} \label{intro}
\subsection{Contexto sobre la detección de exoplanetas}
Desde tiempos ancestrales, la existencia de mundos habitables y posible existencia de vida en ellos ha sido una de las grandes incógnitas del ser humano. Para abordar esta cuestión fundamental, distintas técnicas de detección planetaria han sido desarrolladas por la comunidad científica. Entre ellas, se encuentran el método de velocidad radial \citep{str-52-rad_vel}, basado en el desplazamiento Doppler del espectro estelar; o el método de microlente \citep{mao-91-microlens}, que emplea el efecto de lente gravitacional causado por el sistema binario estrella-planeta. No obstante, la técnica predominante en los últimos años \citep{chr-25-NASA_Exoplanet_Archive} es el método de fotometría de tránsito \citep{bor-84-transit}, en el que se cuantifica la atenuación periódica de flujo estelar por el paso de un exoplaneta. A través de este último método se obtienen razones entre propiedades del planeta y la estrella, haciendo imperativo conocer las características intrínsecas de la estrella anfitriona para determinar la naturaleza del objeto en tránsito. 

\subsection{Objetivos del Estudio}
El presente trabajo se articula sobre una secuencia de hitos metodológicos. Inicialmente, se detallan las propiedades del sistema TOI-4010 y se trata la observación directa e independiente de su estrella. Seguidamente, se describe el proceso de reducción de datos necesario para el cálculo de su magnitud aparente en la banda \textit{V}. A continuación, se presenta el \textit{pipeline} diseñado para la detección de exoplanetas mediante curvas de luz de TESS. Se profundiza en el funcionamiento de las herramientas de \textit{Pixel Level Decorrelation} (PLD), \textit{stitching} y el algoritmo \textit{Box Least Squares} (BLS), que permiten derivar los parámetros físicos $R_p$ y $a$. Una vez validada la metodología con los datos de TESS para los planetas de corto periodo de TOI-4010, se aplica a un conjunto más amplio de exoplanetas y se combina con datos del \textit{NASA Exoplanet Archive}, permitiendo distinguir las composiciones planetarias y evaluar su posible habitabilidad.