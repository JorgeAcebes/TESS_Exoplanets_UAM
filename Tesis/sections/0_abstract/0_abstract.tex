\begin{abstract}
    Este trabajo presenta un marco integral para el estudio de sistemas exoplanetarios combinando detección fotométrica mediante la misión TESS y la caracterización física con datos de archivo. En la fase experimental, se realizó la observación directa de la estrella TOI-4010 determinando su magnitud aparente en la banda \textit{V}, $m_V = 12.1\pm 0.5$ mag. Paralelamente, se desarrolló un \textit{pipeline} computacional basado en la librería \texttt{Lightkurve} para el procesamiento de curvas de luz de TESS, empleando técnicas como PLD, \textit{stitching} multisector y BLS. Este protocolo permitió derivar los parámetros $a, R_p$ del sistema TOI-4010 con alta consistencia con respecto a la literatura, con desviaciones menores de $1\sigma$ en todos los casos. Destaca la determinación del radio de TOI-4010 c, con un error relativo del $1.2\%$ y alcanzando una puntuación $z = 0.17$. La metodología se extendió a una muestra de 339 sistemas monoplanetarios que, junto a la integración de datos del \textit{NASA Exoplanet Archive}, posibilitó un análisis composicional de la población, diferenciando entre planetas rocosos, altos en volátiles y gaseosos; e identificar un candidato en la Zona Habitable: TOI-715 b. Los resultados demuestran la eficacia de combinar fotometría terrestre y espacial con catálogos estelares para una caracterización exoplanetaria robusta, la discriminación de poblaciones planetarias y la localización de mundos potencialmente habitables.
\end{abstract}