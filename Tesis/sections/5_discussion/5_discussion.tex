\section{Discusión}
La determinación experimental de la magnitud aparente de TOI-4010 resultó en $m_{V,\text{exp}}= 12.1 \pm 0.5$ mag, valor consistente con la literatura, $m_{V,\text{lit}} = 12.41 \pm 0.05$ mag, lo que supone una puntuación $z = 0.62$. No obstante, la incertidumbre relativa en la medida de $4.0\%$ evidencia las limitaciones de la observación causadas por el \textit{seeing} y la baja elevación.


En cuanto a los parámetros planetarios derivados para el sistema TOI-4010, estos muestran una excelente consistencia estadística con los valores de referencia (NEA), con todas las medidas distando menos de $1\sigma$ del valor bibliográfico (con una puntuación $z= 0.97$ como máximo, para TOI-4010 d), lo que valida el proceso de extracción fotométrica.


La comparación entre el uso de TPFs y LCF revela que, para el radio planetario, el uso de LCF ofrece un mayor grado de concordancia en promedio, con un error relativo medio $\bar\epsilon_{\text{rel}} = 3.79\%$, frente a TPF $\bar\epsilon_{\text{rel}} = 4.78\%$. En particular, destaca por su exactitud la medida del radio de TOI-4010 c, con $\epsilon_{\text{rel}} = 1.2\%$ y $z=0.17$. En cuanto al semieje mayor, su determinación resulta invariante ante al método de extracción empleado, con $\bar\epsilon_{\text{rel}} = 3.4\%$. Esta dicotomía entre la dependencia de $R_p$ y la independencia de $a$ respecto al origen de los datos reside en los valores proporcionados por el algoritmo BLS. El radio planetario es función de la profundidad de tránsito $\delta$ \eqref{delta_transit}, magnitud extremadamente sensible a procesos de PLD y \textit{detrending}. Por el contrario, el semieje mayor depende del periodo orbital $T$ \eqref{eq_kepler}, magnitud extraída de la periodicidad de la señal, más robusta frente a ruido instrumental y dispersiones en el flujo al tratarse de una característica temporal y no morfológica. Así, este fenómeno se puede atribuir a una limpieza excesiva de las TPFs, que al realizar un sobreajuste elimina contribuciones de la señal de tránsito. Esto es coherente con lo que se aprecia en la Figura \ref{fig:transit_comparison}, una mayor dispersión de puntos en la curva de luz correspondiente a LCF, indicativo de una limpieza no excesiva que preserva la integridad de la señal.

En la ejecución del \textit{pipeline} sobre una población extensa, se obtuvo una eficacia del $84.8\%$, tras excluir aquellos planetas que excedían los límite de detección. Esta tasa de recuperación es destacable, considerando que el proceso se llevo a cabo de forma automatizada y sin una optimización individualizada de los parámetros libres del algoritmo BLS. Esto hace patente la viabilidad de la implementación de búsquedas ciegas automatizadas en grandes catálogos fotométricos. No obstante, es necesario tener en cuenta la alta tasa de error del algoritmo, $10.0\%$, así como el porcentaje de alias detectados, $5.2\%$, corregibles mediante el uso de otros métodos más avanzados como TLS \cite{TLS_aliasing}.

El análisis poblacional muestra resultados reveladores. El diagrama MR (Figura \ref{fig_diagrama_MR}) se ajusta fielmente a lo predicho por Chen \& Kipping y, de acuerdo a su clasificación, se encuentra un predominio de planetas jovianos y saturnianos, habiéndose detectado solo un planeta \textit{terran} (TOI-1776 b). Esto no es representativo de la distribución galáctica real, sino que responde a un sesgo observacional. Puesto que la profundidad del tránsito escala con el cuadrado del radio, $\delta \propto R_p ^2$ \eqref{delta_transit}, los cuerpos de mayor radio generan señales con mayor señal-ruido efectiva $\alpha$, pues $\alpha \propto
\delta$ \eqref{eq:SNR}, favoreciendo la detección de planetas de gran tamaño ($<10\,M_\oplus$). Esta tendencia sistemática provoca que una minoría de los planetas detectados sean planetas rocosos, los cuales se encuentran principalmente entre la línea de agua pura y $100\%$ MgSiO$_3$. La población incrementa para el conjunto de planetas ricos en volátiles, dentro de los que se encuentran los planetas de corto periodo del sistema TOI-4010. El resto de exoplanetas son gigantes gaseosos, los cuales conforman la fracción dominante de la muestra. Por último, hágase notar que aunque se cribaron los periodos no válidos ($\epsilon_{\text{rel}} >0.01$), no se validaron los datos del radio, con lo que este ajuste a los regímenes composicionales teóricos confirma la bondad del modelo.

Finalmente, el análisis de habitabilidad demuestra la dificultad de detección de mundos potencialmente adecuados para la vida, encontrándose solamente TOI-715 b dentro de los límites más conservativos de la Zona Habitable definidos por \citet{zona_habitable},\cite{zona_habitable} y TOI-6002 b cercano al límite de Venus reciente; ambos en concordancia con otros estudios bibliográficos \cite{TOI-715b}, \cite{TOI-6002b}.